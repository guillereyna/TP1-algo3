\documentclass[10pt, a4paper]{article}
\usepackage[paper=a4paper, left=1.5cm, right=1.5cm, bottom=1.5cm, top=3.5cm]{geometry}
\usepackage[spanish]{babel}
\usepackage[utf8]{inputenc}
\usepackage{imakeidx}
\usepackage{caratula}

\begin{document}

\titulo{Informe: Trabajo Práctico 1}
\materia{Algoritmos y Estructuras de Datos III}

% CAMBIAR INTEGRANTES
\integrante{Casado Farall, Joaquin}{072/20}{joakinfarall@gmail.com}
\integrante{Vitali, Lucas}{XXX/XX}{}
\integrante{Reyna Maciel, Guillermo José}{393/20}{guille.j.reyna@gmail.com}
\integrante{Chumacero, Carlos Nehemias}{492/20}{--}
\integrante{Marco}{}{--}

\maketitle

% \tableofcontents

\addcontentsline{toc}{section}{Introducción}
\section*{Introducción}

En este informe vamos a blah blah blah

\addcontentsline{toc}{section}{Problema 1: Clique más influyente}
\section*{Problema 1: Clique más influyente}

Descripcion del problema de la clique mas influyente y sarasa

\section{Backtracking}

\subsection{}
Vamos punto por punto en subsecciones

\section{Podas}
Lo mismo acá

\addcontentsline{toc}{section}{Problema 2: Selección de actividades}
\section*{Problema 2: Selección de actividades}
Descripción del problema de selección de actividades

\section{Programación dinámica}
Solución y demostración

\section{Algoritmo goloso}
Etcéteras

\end{document}
